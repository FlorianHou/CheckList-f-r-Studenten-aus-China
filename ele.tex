\documentclass[doku]{subfiles}
\begin{document}
\begin{itemize}
	\item 手机卡可以提前在淘宝上买个德国沃达丰的手机卡,在淘宝里搜\textbf{德国手机卡}就能搜的到,因为德国这边安装宽带是需要等待安装时间的,德国手机卡的办理也由于疫情只能线上办理,你们初期来德国办理手机卡,我建议先办Vodafone或者Telekom的,这两个属于德国的第一梯队,在网络覆盖不是那么好的德国,这两个的覆盖率还是不错的,再有就是科布伦茨现在公交车上基本都有WiFi了,德铁也逐渐铺开wifi了。\label{simKarte}
	\item 笔记本电脑还是有必要带的,作为理科生,我想说,,苹果入手需谨慎
	\item ipad这样的可以手写的设备,我看朋友用的都还是挺方便的,看自己需需求吧,我还是比较喜欢纸质的感觉。
	\item 手机,国内的手机这里不会有什么太大问题,苹果手机无缝替换,安卓手机能用安装谷歌框架是最好不过的,因为我的手机有欧洲版,所以可以直接安装欧版系统
	\item 充电宝 不用过大,我觉得10000mah的比较平衡
	\item 充电器:按照自己的需求看着带吧
	\item 小音箱:可有可无,和同学一起吃饭时放个歌还是不错的
	\item 耳机:哈哈,不用多说
	\item 计算器: fx-991es是德国这里考试能带的最高等级的计算器
	\item 硬盘: 不要买过大的,500g到1t比较合适
	\item u盘: 16G和32G的也不贵,多带几个吧,丢u盘是我的日常操,我的日常资料基本都在谷歌和onedrive(学生账户有1tb)上有备份
	\item 电池:超市随手就能买
	\item 相机: 有想法想要记录记录生活,可以带一个,不用买的太贵太大的,作为日常用,我还是比较推荐佳能m6,索尼a6400, a7c,这样的比较小巧相机,小巧才有动力带出去,如果手机过于nb,且没兴趣研究相机,请自动忽视这条
	\item 储存卡: 德国这边sd卡不得不说,,还是挺便宜的
	\item 镜头盖: 可以再带几个镜头盖 ,丢了在德国买还是很贵的
	\item 吹风机: 有卖的飞利浦,AEG的20€就能买到,缺点是现在以为疫情商店都还是闭门中,只能线上购买
	\item 手机/平板充电线: 需要的线材都多备份几个
	\item 小手电: 找东西还是挺好用的
	\item 小螺丝刀备一份还是不错的,我用的一直是\textbf{南旗}的
	\item 电源转换器  可以多带几个防止丢失,至少3个吧
	\item 中国插座 我买的是小米带usb接口的插座
	\item 路由器: 我觉得还是有必要带的,例如我的学生宿舍就还需要自己接一个无线路由器,买个100、200的就行
	\item 打印机: 打印机贵,如果包了套餐,墨盒用完后,HP会自动给你寄墨盒,不需要付墨盒钱,感觉还是很划算的,HP和Epson这样的品牌墨盒贵主要是因为,喷头在墨盒上,所以自己买墨盒还是蛮贵的,佳能这样的喷头在打印机上,墨盒就会便宜许多,但是目前如果都是线上上课的话要打印机还是有必要的,在德国这边随便买一个70欧左右的就可以,我之前用的是hp,而且包了套餐,如果包了套餐,墨盒用完后,HP会自动给你寄墨盒,不需要付墨盒钱,感觉还是很划算的,HP和Epson这样的品牌墨盒贵主要是因为,喷头在墨盒上,所以自己买墨盒还是蛮贵的,佳能这样的喷头在打印机上,墨盒就会便宜许多,但是如果喷头坏了,就会很麻烦。HP这样的每次换个墨盒就跟换了个新机器差不多,不用担心喷头老化损坏和堵塞的问题。。
	
\end{itemize}
\end{document}